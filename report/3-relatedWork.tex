\chapter{Related Work}\label{chap:relatedWork}

Connie's Paper?

This chapter reviews the work that is most related to the research questions investigated by you in this work. Please note that there are various options on \emph{where} you include it.

\begin{itemize}
  \item You could include it \emph{here} (i.e., where you see it right now in the template). Since it's after the formal definitions (Chapter~\ref{chap:background}), you can explain what the other works have done on some level of detail, yet you need to keep in mind that you did not yet explain your own contributions (except abstractly in the abstract), which slightly limits the level of technical detail on which you can compare these approaches here.
  
  \item You could also make it a subsection of Chapter~\ref{chap:background}. This choice might also depend on the length of this chapter. Is it worth its own full chapter?

  \item Alternatively, you might include this chapter after the main part of your report, i.e., right before Chapter~\ref{chap:conclusion}. When you do this, you can go into more technical detail since the readers will have read your entire work, so they know exactly what you've done and you can therefore discuss differences (like pros/cons etc.) in more detail.

  \item When you take a look at scientific papers (preferably at top-tier venues), you might notice that not every single paper has a related work section. This is because in principle related works might also be addressed/positioned in the introduction or in the main part of the work. But since this is not a ``standardized scientific publication'', it is very strongly advised that you devote its own section to related work as done in this template.
\end{itemize}

If you prefer any of the latter two options, discuss this with your supervisor(s).
