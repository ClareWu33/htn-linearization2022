\chapter{Introduction}

\section{Introduction to HTN Planning}
High-level introduction (without technical definitions) to research area that can be understood by anybody with some basic mathematical understanding.  

\section{Motivation}

\subsection{overview of potential pros and cons of total vs. partial order}
\begin{itemize}
	\item pros of partial order:
	\begin{itemize}
		\item plan recognition: independent goals can be described in parallel (Daniel should be able to write something about that)
		\item partial order is more expressive (both in terms of plan existence and in terms of computational complexity) meaning that more problems can be expressed
		\item Domain model might be more intuitive: if a task is independent of some others it might be counter-intuitive to demand a certain position of it (if artificially made totally ordered)
	\end{itemize}

   \item pros of total order:
    \begin{itemize}
   		\item computational complexity is lower (fewer worst-case solving time)
        \item we can exploit specialised algorithms as well as heuristics. Note that heuristic design is comparably easy for total-order problems due to the missing interaction between tasks.
   	\end{itemize}
\end{itemize}


\textbf{Further advantages from having a compilation of PO into TO plans}
\begin{itemize}
	\item We actually get another class of decidable partially ordered problems that's orthogonal to tail-recursive ones! (Because if the criterion 'matches', we know that the PO domain is equivalent to the resulting TO domain; the latter is decidable whereas the former is not.) 
	\item We can use more efficient algorithms and heuristics. 
\end{itemize}

\subsection{POCL} 


\section{Contributions} 
Though the algorithm existed and performed well in IPC contest, there does not exist empirical analysis of it's performance, or specification of it's formal properties. This paper provides both.



