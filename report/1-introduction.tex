\chapter{Introduction}

\section{Introduction to HTN Planning}
I have no idea what to write here that isn't already covered somewhere else.

\section{Motivation}
There are several advantages to a totally ordered planning problem:
    \begin{itemize}
   		\item computational complexity is lower (fewer worst-case solving time)
        \item we can exploit specialised algorithms for totally ordered problems, as well as heuristics. Heuristic design is comparably easy for total-order problems due to the missing interaction between tasks.
   	\end{itemize}

Having a compilation of PO into TO plans allows us to apply the advantages of a totally ordered planning problem to partially ordered problems. Also, we get another class of decidable partially ordered problems that's orthogonal to tail-recursive ones. %Because if the cri 

On the other hand, advantages of partial order:
\begin{itemize}
	\item plan recognition: independent goals can be described in parallel (Daniel should be able to write something about that)
	\item partial order is more expressive (both in terms of plan existence and in terms of computational complexity) meaning that more problems can be expressed
	\item Domain model might be more intuitive: if a task is independent of some others it might be counter-intuitive to demand a certain position of it (if artificially made totally ordered)
\end{itemize}
% \subsection{POCL} 

\section{Contributions} 
Though the algorithm was implemented in the old PANDA-3 planner, and was used to produce many of the totally ordered domains in IPC benchmark, there does not exist empirical analysis of it's performance, or specification of it's formal properties. This paper provides both.
It also investigates possible improvements to the original algorithm.
% Dr. Gregor Behnke already implemented a technique for 'this' available in the PANDA-3 planner. It was used to create many IPC TO domains, though it was never published, i.e., neither described nor properties like preserving of solutions were investigated.



