\chapter{Background}\label{chap:background}

\section{Classical planning}
A classical planning problem is usually defined in the STRIPS formalism, as below:
$\textbf{Problem} = (D, S_I, S_G)$ \newline
D is the domain of the problem, \newline
$S_I \in 2^F$ is the initial state and $S_G \in F$ is the goal. \newline
Every fact $\in F$ included in $S_G$ is true, and all other facts are false. This is the closed world assumption.


\textbf{Domain D} = $(F, A)$ \newline
F is a finite set of facts, or propositional state variables. \newline
A is a finite set of actions. \newline Every $a \in A$ is of type $2^F \times 2^F \times 2^F$, and of the format $<pre, add, del>$.
An action a is executable in a state s if its precondition $pre$ holds in s, i.e. pre $\subseteq$ s. \newline
If executable in s, its result is the successor state s' = (s $\backslash$ del ) $\cup$ add , i.e., delete effects get removed and add
effects get added.

\textbf{Solution}: Solutions to a problem are action sequences executable in the initial state $S_I$ that lead to a state s'' that satisfies all
goals, i.e., s'' $ \supseteq$ g. Any such state s'' is called a goal state.

\newpage
\section{Partially Ordered Hierarchical Planning}
Also known as Hierarchical Task Network Planning or HTN planning for short.

There are many formalizations for hierarchical planning. This one borrows heavily from "A Survey on Hierarchical Planning – One Abstract Idea, Many Concrete Realizations" by Bercher, Alford, H\"{o}ller, except with the initial task network replaced by a initial compound task. 

\textbf{Problem} = $(D, S_I, T_I)$
is over some domain D, 
has an initial state $S_I$, which is a total assignment to F, and 
has a initial compound task $T_I$. 

\textbf{D} = $(F, T_P, T_C, \delta, M)$ \newline
   F is the finite set of state variables, $T_P$ is the finite set of primitive task names, $T_C$ is the finite set of compound task names, and
   $\delta$ is a mapping from primitive task name to action.
   M is the finite set of decomposition methods. Each one maps a compound task name to a task network.
   $T_P$ is the set of all possible primitive task names
   $T_C$ is the set of all possible compound task names
   $\delta$ maps actions to it's (preconditions, add, deletes) $\in 2^F x\times 2^F \times 2^F$. This can alternatively be referred to as $prec(t), add(t), del(t)$ for $a \in F, t \in T_P$
   M is a set of methods. If $m \in M$, then m maps a compound task to a Task Network.

\textbf{Task Network} = $(T, \prec, \alpha)$
 T is a finite set of task ids,
 $\prec$ is a partial order over T, and
 $\alpha$ maps task ids to task names.
 
\subsection{HTN Solution}
The goal here is to refine this task network, such that a valid solution is obtained.
A valid solution to a PO HTN problem a plan:
   - a sequence of primitive tasks/actions
   - The application of each action produces a corresponding state. The solution must be such that each action is applicable
    in the previous state. i.e. the corresponding state satisfies the precondition of the action (including the first action applicable in the initial state)
   - (Often, the final state produced will be checked against a 'goal state' encoded in a (artificially enforced) final primitive tasks' preconditions,? 
     
     
\section{Totally Ordered Hierarchical Planning}
Totally ordered hierarchical planning is the same as partially ordered planning in all respects except the task network.
Both define a \textbf{Task Network} = $(T, \prec, \alpha)$.
The difference is that $\prec$ now specifies the order between task ids such that for every a, b, in T, there exists an ordering
in $\prec$ that $(a, b)$ or $(b, a)$.
In other words, the tasks are totally ordered. 


\section{Further Definitions}
\textbf{Undecidable } A decision problem is equivalent to, a function that accepts a infinite (set?) of inputs and returns a yes or no. The decision problem is undecidable if it can be proved that there exists no algorithm for the problem that always leads to a correct yes-or-no answer.

%\textbf{Task Decomposition Graph}
%Let $G = (V_T, V_M , E)$ be a TDG in accordance to Def. 5
%given by Elkawkagy et al. (2012). 

\textbf{Task Decomposition Tree}
Given a planning problem P, then a decomposition tree g = (T, E, $\prec$, $\alpha$, $\beta$) is a 5-tuple with the following properties. (T,E) is a tree with nodes T and directed edges E pointing towards the leafs. There is a strict partial order defined over the nodes, given by $\prec$. The nodes are labeled with task names by $\alpha$ : T $\rightarrow$ C $\cup$ O.  Additionally, $\beta$ : T $\rightarrow$ M labels inner inner nodes with methods.
$T(g)$ refers to the tasks of $g$, and $ch(g,t)$ refers to the direct children of $t \in T(g)$ in $g$. 
 

\textbf{Lifted}
Each object that exists in the world has a type from a pre-determined set of types defined by the domain.
Methods apply to the set of object(s) that belong to a certain type.

\textbf{Grounded}
A transition is ground if the parameters list only involves specific objects. 
Problems are grounded when all methods are grounded.

So to ground a problem: Let $\omega$ be a set of typed objects. The groundings of
a transition schema a over $\omega$  is denoted by $\sigma$(a, $\omega$ ) and corresponds to the set of all ground transitions obtained by
substituting $\sigma$ with a list of compatible objects taken from $\omega$ , and then substituting each occurrence of the variables
which were in $\sigma$ with the newly introduced objects.  

