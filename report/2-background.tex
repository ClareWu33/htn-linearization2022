\chapter{Background}\label{chap:background}

\section{Hierarchical Planning}
We first introduce classical planning, as hierarchical
planning can be considered an extension of classical planning.

\subsection{Planning Problem Formalisation}
Classical planning problems are defined over a \textbf{domain D} = $(F, A)$, where 
$F$ is a finite set of facts, or propositional state variables.
$A$ is a finite set of actions. 
For all $a \in A$,  $a \in 2^F \times 2^F \times 2^F$, which represents the preconditions and add and delete effects of an action. %$\langle pre, add, del \rangle$. 
The preconditions, add, and delete effects, of an action $a$ are referred to as $\Pre(a), \Add(a)$, and $\Del(a)$ respectively.
An action $a$ is executable in a state $s$ if its precondition $pre$ holds in $s$, i.e. $pre \subseteq s$. 
If executable in $s$, its result is the successor state $s'$ = $(s \backslash \Del) \cup \Add$, i.e., variables in $\Del(a)$ get removed and variables in $\Add(a)$ get added.

A \textbf{classical planning problem} is defined as $\textbf{P} = (D, S_I, S_G)$, 
where $D$ is the domain of the problem, 
$S_I \in 2^F$ is the initial state and $S_G \in F$ is the goal description.
Given a state $s \in 2^F$ , a fact is known for certain to be either true or false, i.e. information about the world state is complete at all times (this is called the closed world assumption).

A \textbf{solution} $\bar{a}$ to a problem is any action sequence that is executable in the initial state $S_I$, and leads to a state $s''$ that satisfies all
goals, i.e., $s'' \supseteq S_G$. Any such state $s''$ is called a goal state.


There are many formalisations for hierarchical planning, also known as \textbf{HTN planning}. The following one borrows heavily from \cite{HTNSurvey} and \cite{Geier2011TIHTNDecidability}. HTN planning has two variants, Partially Ordered Hierarchical Task Network Planning, also known as POHTN planning, and Totally Ordered Hierarchical Planning, also called TOHTN planning,   % except with the initial task network replaced by a initial compound task. 

A POHTN problem \textbf{P} = $(D, S_I, T_I)$
is defined over some domain $D$, 
has an initial state $S_I \in 2^F$, and 
has a initial compound task $T_I$. 
The closed world assumption also holds for HTN planning.

The domain \textbf{D} = $(F, T_P, T_C, \delta, M)$, where
$F$ is the finite set of facts or state variables, $T_P$ is the finite set of all possible primitive task names, $T_C$ is the finite set of all possible compound task names, and
$\delta$ is a mapping from primitive task name to action. Actions in POHTN domain also have preconditions, adds, and delete effects.
$M$ is the finite set of decomposition methods. Each one maps a compound task name to a task network. If $m \in M$, then $m=(c, tn)$, where $c \in T_C$.

A task network \textbf{tn} = $(T, \singlePrec, \alpha)$ consists of
T, which is a finite set of task identifiers (ids); 
$\singlePrec$, which is a partial order over T; and
$\alpha$, which maps task ids $\in$ T to task names in $T_C$ and $T_P$. 

% replacing t  (i.e. $tn_1 \rightarrow_{t,m} tn_2$)
A method $m$ decomposes a task network $tn_1 = (T_1, \singlePrec_1, \alpha_1)$ into
a new task network $tn_2$ by replacing $t$, if and only if $t \in T_1$, $\alpha_1(t) = c$, and $\exists tn' = (T', \singlePrec' \alpha')$ with $tn' \cong tn_m$ and $T' \cap T = \emptyset$ and
\begin{align}
tn_2 :=     &((T_1 \setminus \{t\}) \cup T',    \singlePrec_1 \cup \singlePrec' \cup \singlePrec_X,        \alpha_1 \cup \alpha') \\
\singlePrec_X :=  &\{(t_1, t_2) \in T_1 \times T'  \mid  (t_1,t) \in \singlePrec_1 \} \cup \\
&\{(t_1, t_2) \in T' \times T_1 \mid (t, t_2) \in \singlePrec_1 \}  
\end{align}
In other words, the decomposition of a compound task results in it being removed from the task network and replaced by a copy of the method’s task network. The ordering constraints
on the removed task are inherited by its replacement tasks, as defined by $\singlePrec_X$. 

A solution to an HTN problem is a task network $tn = (T, \prec, \alpha)$ if and only if
tn can be reached via decomposing $tn_I$, all tasks are primitive, ($\forall t \in T: \alpha(t) \in T_P$), and there exists a sequence $\langle t_1, t_2 ... t_n \rangle$ of the task ids in $T$ that agrees with $\prec$ such that the application of that sequence $\langle \alpha(t_1), \alpha(t_2) ... \alpha(t_n) \rangle$ in $S_I$ is executable.

In other words, the goal of hierarchical planning is to find an decomposition of the task, then any executable refinement of the resulting decomposition. Whereas in classical planning, one only finds any executable sequence of actions to achieve a goal state, so HTN planning poses additional restrictions on which action sequences may be considered.

\textbf{Totally Ordered Hierarchical Planning}, also called TOHTN planning, is the same as partially ordered planning in all respects except the kind of task networks it allows.
For both planning formalisms, a method $m$ maps a task $t$ to a task network \textbf{tn} = $(T, \prec, \alpha)$. TO planning domains require that $\prec$ must specify a total order between task ids in $T$.
This leads to a difference in expressiveness and decide-ability of TOHTN vs POHTN planning. POHTN planning is more expressive in general (both in terms of plan existence and in terms of computational complexity). As per \cite{LanguageClassificationPlanning}, if regarded from the standpoint of formal grammars, TO planning is exactly as expressive as context free languages, whereas PO planning is strictly more expressive than context-free languages, and strictly less expressive than context-sensitive languages.
In terms of complexity classes, \cite{ErolHTNExpressivity} proved that POHTN planning is semi-decidable, whereas \cite{Alford2015TightHTNBounds} proved that, assuming arbitrary recursion, TOHTN planning is 2-EXPTIME-complete with variables, and EXPTIME-complete without. 


%\textbf{Tail-recursive problems} are HTN problems where methods can only recurse through the last task in their task network. Problems that don't have recursion are also tail-recursive. \cite{Alford2015TightHTNBounds}

%\todo{maybe rephrase it a little bit? there are a lot of stuffs between decidable and undecidable problems}% Relying on a comparison to the Chomsky hierarchy, H\"{o}ller et al. [2014;2016] conducted an in-depth  showing which problems (i.e., which intended solution sets) can be expressed by various hierarchical and non-hierarchical formalisms. 

%\subsubsection{Grammars}
%The following explanation for \cite{ChomskyHeirarchyRogers}
%A grammar $G$ is defined as $(N, \epsilon, P, S)$, where
%$N$ is  a set of non-terminal symbols, 
%$\epsilon$ is a set of terminal symbols,
%$P$ a set of production rules, that maps elements in $N$ to a finite sequence of elements in %$\epsilon$
%$S$ is the start symbol.
%A $Language$, defined by a specific grammar $G$, consists of all strings that can be produced by repeated application, including infinite applications, of the production rules to the start symbol $S$.

%In a context-free grammar, all production rules $\in P$ take the form $A \implies b$ where $A$ is a single non-terminal symbol and $b$ is a string of symbols. CFLs are those languages that %can be defined by a context-free grammar.

%The Chomsky hierarchy defined a relationship between these different classes of grammars, based off whether one grammar can pruduce all possible languages possible using another grammar class. Context-sensitive grammars are more general than context-free grammars, in the sense that there are languages that can be described by CSG but not by context-free grammars.


\section{Relevant Properties for Planning Classes}
A decision problem is equivalent to a function that accepts (potentially infinite) inputs and returns a yes or no. The decision problem is \textbf{undecidable} if it can be proved that there exists no single algorithm for the problem that always leads to a correct yes-or-no answer. Partially ordered HTN planning is undecidable, in that
there is no algorithm that can determine whether or not an arbitrary partially ordered problem has a solution. while totally ordered HTN planning is decidable.


%\textbf{Task Decomposition Graph}
%Let $G = (V_T, V_M , E)$ be a TDG in accordance to Def. 5
%given by Elkawkagy et al. (2012). 

%\textbf{Task Decomposition Tree}
%Given a planning problem P, then a decomposition tree $G$ = (T, E, $\prec$, $\alpha$, $\beta$). (T,E) is a tree with nodes T and directed edges E %pointing towards the leafs. There is a strict partial order defined over the nodes, given by $\prec$. The nodes are labeled with task names by $\alpha$ : T $\rightarrow$ C $\cup$ O.  Additionally, $\beta$ : T $\rightarrow$ M labels inner inner nodes with methods.
%$T(g)$ refers to the tasks of $g$, and $ch(g,t)$ refers to the direct children of $t \in T(g)$ in $g$. 


HTN problems can be modelled in a \textbf{lifted} way. Each object that exists in the world has a type from a pre-determined set of types defined by the domain.
The parameters of the method apply to the set of object(s) that belong to a certain type.

HTN problems can also be modelled in a \textbf{grounded} way. A method is ground if the parameters list only involves specific objects. 
Problems are grounded when all methods are grounded. 

We can ground a lifted problem in polynomial-time. Assuming $\omega$ is a set of typed objects in a lifted problem, the groundings of
a transition schema $a$ over $\omega$  is denoted by $\sigma(a, \omega)$ and corresponds to the set of all ground methods obtained by
substituting $\sigma$ with a list of compatible objects taken from $\omega$ , and then substituting each occurrence of the variables
which were in $\sigma$ with the newly introduced objects.  

\textbf{Partial Order Causal Link Planning} is a planning technique that only binds variables necessary to reach the current sub-goals.

A partial POCL plan is a tuple $P = (PS, \singlePrec, CL)$, where 
$PS$ is a finite set of plan steps $ps = (l, a)$ with $l$ being a label
unique in $PS$, $a \in A$ an action; $\singlePrec$ is a partial order on $PS$;
and $CL$ is a finite set of causal links. 

A causal link $cl = (ps_p, v, ps_c) \in CL$ indicates that the precondition $v$ of the consumer plan step $ps_c$ is
supported by the producer plan step $ps_p$ (i.e., it also implies
$v \in prec(ps_c) \cap add (ps_p)$.


A causal threat is the situation where a plan step $ps_t$ with $v \in \Del (ps_t)$ may be ordered between $ps_p$ and $ps_c$, i.e., the ordering constraints may neither entail $ps_t \prec ps_p$ nor $ps_c \prec ps_t$.

A partial POCL plan $P = (PS , \singlePrec, CL)$ is called POCL solution to a planning problem if and only if every precondition is supported by a causal link and there are no causal threats. The solution criteria for POCL plans can obviously be checked in lower polynomial time.


% From Bercher \& Olz, AAAI-2020 POPOCL 
%\emph{POCL criterion.} A partial POCL plan P = (PS, $\prec$, CL) is called POCL plan (also POCL solution) to a planning problem if and only if every precondition is supported by a causal link and there are no causal threats.

%\emph{PO criterion.} We refer to a partial POCL plan P = (PS , $\prec$, CL) without causal links, i.e., CL = $\emptyset$, as a partial partially ordered (PO) plan. Due to the absence of causal links, the solution criteria are here defined directly by the desired property of every linearization being executable.

%i.e. A partial PO plan P = (PS , $\prec$) is called PO plan (also PO solution) to a planning problem if and only
%if every linearization is executable in the initial state and results into a goal state.
%i.e. it has every necessary ordering. Therefore a linearization exists if and only if a POCL solution exists.